\documentclass[12pt]{report}

%%%%%%%%%%%%%%%%%%%%%%%%%%%%%%%%%%%%%%%%%%%%%%%%%%%%%%%%%%%%%%%%%%%%
% package and document formatting stuff
%%%%%%%%%%%%%%%%%%%%%%%%%%%%%%%%%%%%%%%%%%%%%%%%%%%%%%%%%%%%%%%%%%%%

% symbols and math stuff
\usepackage{amsmath,amsthm,amssymb}
\usepackage{tikz}
\usepackage{tkz-graph}
\usetikzlibrary{arrows,%
                petri,%
                topaths}%

\usepackage{tkz-berge}


% math operators
\usepackage{amsopn}

% script and caligraphics
\usepackage{eucal,mathrsfs}

% indexing
\usepackage{makeidx}

\usepackage{enumerate}

% formatting
\usepackage{fullpage}

% links and colors
\usepackage{color}
\usepackage[pdfstartview=FitH,
%             pdfauthor={\myauthor},
%             pdftitle={\mytitle},
            colorlinks,
            linkcolor=reference,
            citecolor=citation,
            urlcolor=e-mail,
            backref]{hyperref}
\usepackage[all]{xy}

\definecolor{todo}{rgb}{.80,.20,.20}
\definecolor{e-mail}{rgb}{0,.40,.80}
\definecolor{reference}{rgb}{.10,.40,.42}
\definecolor{mrnumber}{rgb}{.80,.40,0}
\definecolor{citation}{rgb}{0,.40,.80}

%%%%%%%%%%%%%%%%%%%%%%%%%%%%%%%%%%%%%%%%%%%%%%%%%%%%%%%%%%%%%%%%%%%%
% theorem stuff
%%%%%%%%%%%%%%%%%%%%%%%%%%%%%%%%%%%%%%%%%%%%%%%%%%%%%%%%%%%%%%%%%%%%

\theoremstyle{plain}

\newtheorem{thm}{Theorem}[section]
\newtheorem{defn}[thm]{Definition}
\newtheorem{deflem}[thm]{Definition/Lemma}
\newtheorem{notn}[thm]{Notation}
\newtheorem{convention}[thm]{Convention}
\newtheorem{lem}[thm]{Lemma}
\newtheorem{aside}[thm]{Aside}
\newtheorem{rem}[thm]{Remark}
\newtheorem{ex}[thm]{Example}
\newtheorem{facts}[thm]{Facts}
\newtheorem{cor}[thm]{Corollary}
\newtheorem{conj}[thm]{Conjecture}
\newtheorem{prop}[thm]{Proposition}

\newtheorem{question}[thm]{Question}
\newtheorem{exercise}{Exercise}[section]

%%%%%%%%%%%%%%%%%%%%%%%%%%%%%%%%%%%%%%%%%%%%%%%%%%%%%%%%%%%%%%%%%%%%
% typography stuff
%%%%%%%%%%%%%%%%%%%%%%%%%%%%%%%%%%%%%%%%%%%%%%%%%%%%%%%%%%%%%%%%%%%%

\newcommand{\mb}[1]{\mathbf #1}
\newcommand{\mbb}[1]{\mathbb #1}
\newcommand{\mf}[1]{\mathfrak #1}
\newcommand{\mc}[1]{\mathcal #1}
\newcommand{\ms}[1]{\mathscr #1}
\newcommand{\mcu}[1]{\mathcu #1}
\newcommand{\oper}[1]{\operatorname{#1}}

\newcommand{\da}{\downarrow}
\newcommand{\ra}{\rightarrow}
\newcommand{\hra}{\hookrightarrow}
\newcommand{\dra}{\dashrightarrow}
\newcommand{\la}{\leftarrow}
\newcommand{\lra}{\longrightarrow}

\newcommand{\ov}{\overline}
\newcommand{\til}{\widetilde}
\newcommand{\wh}{\widehat}

\newcommand{\ZZ}{\mathbb{Z}}

\newcommand{\lcm}{\oper{lcm}}

%%%%%%%%%%%%%%%%%%%%%%%%%%%%%%%%%%%%%%%%%%%%%%%%%%%%%%%%%%%%%%%%%%%%
% other stuff
%%%%%%%%%%%%%%%%%%%%%%%%%%%%%%%%%%%%%%%%%%%%%%%%%%%%%%%%%%%%%%%%%%%%

\newcommand{\todo}[1]{\textcolor{todo}{#1}}

%%%%%%%%%%%%%%%%%%%%%%%%%%%%%%%%%%%%%%%%%%%%%%%%%%%%%%%%%%%%%%%%%%%%
% end preamble
%%%%%%%%%%%%%%%%%%%%%%%%%%%%%%%%%%%%%%%%%%%%%%%%%%%%%%%%%%%%%%%%%%%%

\begin{document}

%%%%%%%%%%%%%%%%%%%%%%%%%%%%%%%%%%%%%%%%%%%%%%%%%%%%%%%%%%%%%%%%%%%%
% title stuff
%%%%%%%%%%%%%%%%%%%%%%%%%%%%%%%%%%%%%%%%%%%%%%%%%%%%%%%%%%%%%%%%%%%%


\author{Daniel Krashen}
\title{Notes on Signal Processing\\Discrete Fourier and Wavelet Transforms}
\date{\today}

\maketitle
\tableofcontents

%%%%%%%%%%%%%%%%%%%%%%%%%%%%%%%%%%%%%%%%%%%%%%%%%%%%%%%%%%%%%%%%%%%%
% document stuff
%%%%%%%%%%%%%%%%%%%%%%%%%%%%%%%%%%%%%%%%%%%%%%%%%%%%%%%%%%%%%%%%%%%%

\iffalse
\chapter*{Version Notes}
\addcontentsline{toc}{chapter}{Version Notes}
\fi

\iffalse
\section{Updates 1/25/2020}
\begin{itemize}
\item Nothing to say, but generally update information would go here.
\item e.g. added stuff to Section~\ref{lecture 1}
\end{itemize}
\fi

\chapter{The Discrete Fourier Transform}

The discrete Fourier transform is a tool to go between describing a sampled, periodic signal as a list of measurements and describing it as a combination of basic waveforms.

\section{Periodic signals and their representations}

\subsection{Periodic signals as vectors}

We start by imagining we have a periodic signal $f$, which we sample at a times $0, 1, 2, N-1$. We denote these measurements as
\[ f[0], f[1], f[2], \ldots, f[N-1]. \]
We assume that the signal is periodic with period $N$, so that the value at $N$ agrees with the value at $0$, the value at $N+1$ agrees with the value at $1$, etc. That is, we have $f[N] = f[0]$, and more generally $f[k + N] = f[k]$ for all $k$.

We may write this information as a vector
\[ 
\vec f = 
\left[
\begin{matrix}
f[0] \\
f[1] \\
\vdots \\
f[N-1]	
\end{matrix}
\right]
\]

That is, if we write $f$ for the sampled period signal, we will write $\vec f$ for the vector whose coordinates are the values sampled.
Implicitly, in doing this, we are regarding these (sampled periodic) signals themselves as forming a vector space, and we have chosen a basis of such signals. Explicitly, when we write

\begin{align*}
\vec f = 
\left[
\begin{matrix}
f[0] \\
f[1] \\
\vdots \\
f[N-1]	
\end{matrix}
\right]
&=
f[0]
\left[
\begin{matrix}
1 \\
0 \\
\vdots \\
0 \\
0	
\end{matrix}
\right]
+
f[1]
\left[
\begin{matrix}
0 \\
1 \\
\vdots \\
0 \\
0	
\end{matrix}
\right]
+ 
\cdots
+
f[N-1]
\left[
\begin{matrix}
0 \\
0 \\
\vdots \\
0 \\
1
\end{matrix}
\right] \\
\end{align*}
we think of this as corresponding to the equation
\[f = f[0]e_0 + f[1] e_1 + \cdots f[N-1]e_{N-1}\\
= \sum_{k = 0}^{N-1} f[k] e_k,
\]
where the signal $e_k$ has values described by
\[ e_k[j] = 0 \text{ if $k \neq j$, and } e_k[k] = 1. \]

\subsection{The wave basis for period signals}

Let us introduce the basic ``wave functions'' which we will use to describe our signals. These special periodic signals are described as complex number of unit length, that rotate about the origin at a constant rate.

That is, such a wave function $E$ will be determined by an angle $\theta$, and when measured at time $j$, yields a complex number of unit length, making an angle of $j\theta$ in the counterclockwise direction with the positive $x$-axis, as shown below.

As described in Chapter~\ref{complex exponentials}, we may represent this complex number as 
\[E[j] = e^{ij\theta} = \cos{j\theta} + i \sin{j \theta}.\]



\section{Filters, circular convolution and circulant matrices}

\section{The fast Fourier transform}

\appendix

\chapter{Complex exponentials}

\section{Complex arithmetic and polar form}

Although in typical applications, we deal with real numbers (to some fixed precision), allowing ourselves to work also with complex number can add a great variety of additional tools to bring to bear. In particular, we will eventually see that periodic functions, such as the sine and cosine, can be efficiently and compactly described via complex exponential functions, and doing this makes it a great deal more efficient decompose periodic signals into basic waveforms (i.e. via the Fourier transform).

Recall that a complex number $z = x + iy$, can be thought of graphically as a point in the complex plane, where we think of $x$ and $y$ as the coordinates of a point $(x, y)$. In this representation, it is clear that addition of complex numbers corresponds to addition of vectors -- that is
\[ (x_1 + i y_1) + (x_2 + i y_2) = (x_1 + x_2) + i (y_1 + y_2)\]
agrees with the vector sum
\[ (x_1, y_1) + (x_2, y_2) = (x_1 + x_2, y_1 + y_2) \]
But what about complex multiplication? For multiplication we see a nice visual interpretation by considering a polar representation of the complex number: if the point $(x_j, y_j)$, $j = 1, 2$ in the plane correpsonds to a vector of length $r_j$ and making an angle of $\theta_j$ with the positive $x$-axis, then we may write
\[z_j = x_j + i y_j = r_j \cos \theta_j + i r_j \sin \theta_j \]
and we find (using the sum formula for sine and cosine):

\begin{align} \label{angles add}
(x_1 + i y_1)(x_2 + i y_2) &= (r_1 \cos \theta_1 + i r_1 \sin \theta_1)(r_2 \cos \theta_2 + i r_2 \sin \theta_2) \\ &= r_1 r_2 \Big( (\cos \theta_1 \cos \theta_2 - \sin \theta_1 \sin \theta_2) (\cos \theta_1 \sin \theta_2 + \cos \theta_2 \sin \theta_1 )\Big) \\ &= r_1 r_2 \cos (\theta_1 + \theta_2) + i r_1 r_2 \sin (\theta_1 + \theta_2).
\end{align}

Said another way, we can visualize the multiplication of complex numbers in the following way: \textbf{lengths multiply and angles add}.

\section{Complex Exponentials} \label{complex exponentials}

We can view complex multiplication in a more intuitive light with the aid of Euler's formula, which gives a relationship between the complex exponential function and the polar representation of a complex number. Euler's formula states that for a real number\footnote{in fact the formula holds for complex numbers as well, and can be taken as a definition of cosine and sine for complex arguments} $\theta$, we have
\[e^{i \theta} = \cos \theta + i \sin \theta. \]
This formula can be verified by, for example, replacing each side by the corresponding power series expansions.

The complex exponential function -- that is, the function $e^z$ where $z$ can be any complex number, shares the usual familar properties with its real counterpart: $e^{z + w} = e^z e^w$, immediately giving again the multiplication rule from equation~\ref{angles add} above.

\iffalse
Let's visualize how the graph of the function $e^{i \theta}$ as a function of the real variable $\theta$. 
\fi

\iffalse
Given a complex number, expressed as \[z = r \cos \theta + i r \sin \theta,\] we see that we can raise $z$ to positive integer powers by the formula
\[z^n = r^n \cos n\theta + i r^n \sin n\theta.\]
In particular, it makes sense to extend this definition and simply define the exponential $z^s$ for any real number $s$, by the formula
\[z^s = r^s \cos s\theta + i r^s \sin s\theta\]
\fi

\section{Roots of unity}

%\bibliographystyle{alpha}
%\bibliography{citations}
\printindex

\end{document}
